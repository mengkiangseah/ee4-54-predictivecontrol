\documentclass[letterpaper, 10 pt, conference]{ieeeconf} % Comment this line out if you need a4paper

% \documentclass[a4paper, 10pt, conference]{ieeeconf}   % Use this line for a4 paper

\overrideIEEEmargins                   % Needed to meet printer requirements.

% See the \addtolength command later in the file to balance the column lengths
% on the last page of the document

% The following packages can be found on http:\\www.ctan.org
\usepackage{graphics} % for pdf, bitmapped graphics files
%\usepackage{epsfig} % for postscript graphics files
%\usepackage{mathptmx} % assumes new font selection scheme installed
%\usepackage{times} % assumes new font selection scheme installed
\usepackage{amsmath} % assumes amsmath package installed
\usepackage{amssymb} % assumes amsmath package installed

\title{\LARGE \bf
Meng Kiang SEAH, CID: 00699092
}

\author{EE4-54: Predictive Control. Assignment 5. % <-this % stops a space
}


\begin{document}

\maketitle
\thispagestyle{empty}
\pagestyle{empty}


%%%%%%%%%%%%%%%%%%%%%%%%%%%%%%%%%%%%%%%%%%%%%%%%%%%%%%%%%%%%%%%%%%%%%%%%%%%%%%%%
\begin{abstract}
Discussion of the results of Part A and B, and the conclusions drawn. Perhaps a little bit on the methods employed as well.
\end{abstract}


%%%%%%%%%%%%%%%%%%%%%%%%%%%%%%%%%%%%%%%%%%%%%%%%%%%%%%%%%%%%%%%%%%%%%%%%%%%%%%%%
\section{Part A - Tracking a Square}
\subsection{Algorithm}
\subsection{Evaluation of Performance}
\subsection{Constraind vs. Unconstrained}

This template provides authors with most of the formatting specifications needed for preparing electronic versions of their papers. All standard paper components have been specified for three reasons: (1) ease of use when formatting individual papers, (2) automatic compliance to electronic requirements that facilitate the concurrent or later production of electronic products, and (3) conformity of style throughout a conference proceedings. Margins, column widths, line spacing, and type styles are built-in; examples of the type styles are provided throughout this document and are identified in italic type, within parentheses, following the example. Some components, such as multi-leveled equations, graphics, and tables are not prescribed, although the various table text styles are provided. The formatter will need to create these components, incorporating the applicable criteria that follow.

\section{Part B - Q2: Tracking Non-Square Paths}


\begin{itemize}

    \item Use either SI (MKS) or CGS as primary units. (SI units are encouraged.) English units may be used as secondary units (in parentheses). An exception would be the use of English units as identifiers in trade, such as 3.5-inch disk drive .
    \item Avoid combining SI and CGS units, such as current in amperes and magnetic field in oersteds. This often leads to confusion because equations do not balance dimensionally. If you must use mixed units, clearly state the units for each quantity that you use in an equation.

\end{itemize}

\begin{align}
    J = \frac{1}{N} \sum_{n=1}^{N} \lvert \lvert x_n - \widetilde{x_n} \rvert \rvert ^2 = \sum_{i = M+1}^{D} \lambda_i \label{eq:erroreq}
\end{align}

Just a little check, Equation \ref{eq:erroreq}.

\subsection{Some Common Mistakes}
\begin{itemize}

    \item The word data is plural, not singular.
    \item A graph within a graph is an inset , not an insert . The word alternatively is preferred to the word alternately (unless you really mean something that alternates).
    \item Do not use the word essentially to mean approximately or effectively .
    \item In your paper title, if the words that uses can accurately replace the word using , capitalize the u ; if not, keep using lower-cased.
    \item Be aware of the different meanings of the homophones affect and effect , complement and compliment , discreet and discrete , principal and principle .
    \item Do not confuse imply and infer.

\end{itemize}

\subsection{Figures and Tables}

Positioning Figures and Tables: Place figures and tables at the top and bottom of columns. Avoid placing them in the middle of columns. Large figures and tables may span across both columns. Figure captions should be below the figures; table heads should appear above the tables. Insert figures and tables after they are cited in the text. Use the abbreviation Fig. 1 , even at the beginning of a sentence \cite{IEEEexample:shellCTANpage} \cite{IEEEexample:IEEEwebsite}.

\begin{table}[h]
\caption{An Example of a Table}
\label{table_example}
\begin{center}
\begin{tabular}{|c||c|}
\hline
One & Two\\
\hline
Three & Four\\
\hline
\end{tabular}
\end{center}
\end{table}


  \begin{figure}[thpb]
   \centering
   \framebox{\parbox{3in}{We suggest that you use a text box to insert a graphic (which is ideally a 300 dpi TIFF or EPS file, with all fonts embedded) because, in an document, this method is somewhat more stable than directly inserting a picture.
}}
   %\includegraphics[scale=1.0]{figurefile}
   \caption{Inductance of oscillation winding on amorphous
    magnetic core versus DC bias magnetic field}
   \label{figurelabel}
  \end{figure}

\section{Conclusion}

A conclusion section is not required. Although a conclusion may review the main points of the paper, do not replicate the abstract as the conclusion. A conclusion might elaborate on the importance of the work or suggest applications and extensions.

\bibliographystyle{IEEEtran}
\bibliography{final}

% \addtolength{\textheight}{-12cm}  % This command serves to balance the column lengths
                 % on the last page of the document manually. It shortens
                 % the textheight of the last page by a suitable amount.
                 % This command does not take effect until the next page
                 % so it should come on the page before the last. Make
                 % sure that you do not shorten the textheight too much.

%%%%%%%%%%%%%%%%%%%%%%%%%%%%%%%%%%%%%%%%%%%%%%%%%%%%%%%%%%%%%%%%%%%%%%%%%%%%%%%%



\end{document}
